\documentclass[$if(fontsize)$$fontsize$,$endif$$if(lang)$$babel-lang$,$endif$$if(papersize)$$papersize$paper,$endif$$for(classoption)$$classoption$$sep$,$endfor$]{$documentclass$}

\usepackage{booktabs}
\usepackage{longtable}
\usepackage{framed,color}
\definecolor{shadecolor}{RGB}{248,248,248}

\makeatletter
\newenvironment{kframe}{%
\medskip{}
\setlength{\fboxsep}{.8em}
 \def\at@end@of@kframe{}%
 \ifinner\ifhmode%
  \def\at@end@of@kframe{\end{minipage}}%
  \begin{minipage}{\columnwidth}%
 \fi\fi%
 \def\FrameCommand##1{\hskip\@totalleftmargin \hskip-\fboxsep
 \colorbox{shadecolor}{##1}\hskip-\fboxsep
     % There is no \\@totalrightmargin, so:
     \hskip-\linewidth \hskip-\@totalleftmargin \hskip\columnwidth}%
 \MakeFramed {\advance\hsize-\width
   \@totalleftmargin\z@ \linewidth\hsize
   \@setminipage}}%
 {\par\unskip\endMakeFramed%
 \at@end@of@kframe}
\makeatother
\newenvironment{Shaded}{\begin{kframe}}{\end{kframe}}

\makeatletter
\@ifundefined{Shaded}{
}{\renewenvironment{Shaded}{\begin{kframe}}{\end{kframe}}}
\makeatother
$for(bibliography)$
\addbibresource{$bibliography$}
$endfor$

\providecommand{\tightlist}{%
    \setlength{\itemsep}{0pt}\setlength{\parskip}{0pt}}

$if(href2footnote)$% 是否把链接改为脚注
\let\oldhref\href
\renewcommand{\href}[2]{#2\footnote{\url{#1}}}
$endif$

\usepackage{listings}
% https://github.com/jgm/pandoc/issues/4716
% lstinline 对数学模式存在 bug,等修复后可以改为如下方式
% \newcommand{\passthrough}[1]{#1}
\newcommand{\passthrough}[1]{\lstset{mathescape=false}#1\lstset{mathescape=true}} 


% 感谢 AlexaraWu 的指导 https://github.com/sjtug/SJTUThesis/issues/343
% 兼容 Bookdown 定理等框架
% 重新定义定理环境
\usepackage{amsthm}
\makeatletter
\theoremstyle{plain}
  \newtheorem{theorem}{\sjtu@label@thm~}[chapter]
  \newtheorem{lemma}[theorem]{\sjtu@label@lem~}
  \newtheorem{proposition}[theorem]{\sjtu@label@prop~}
  \newtheorem{corollary}[theorem]{\sjtu@label@cor~}
\theoremstyle{definition}
  \newtheorem{definition}{\sjtu@label@defn~}[chapter]
  \newtheorem{exercise}{练习}[chapter]
  \newtheorem{conjecture}{\sjtu@label@conj~}[chapter]
  \newtheorem{example}{\sjtu@label@exmp~}[chapter]
\theoremstyle{remark}
  \newtheorem{remark}{\sjtu@label@rem~}
  \newtheorem*{solution}{\bfseries{解答}}
\makeatother

% 添加引言块
\def\VA#1#2{\addvspace{12pt}\raggedleft #1\rightskip3em\par #2\rightskip3em}
\renewenvironment{quote}
  {\list{}{\rightmargin\leftmargin}%
    \item\relax}
  {\endlist}

% https://tex.stackexchange.com/questions/154570/itemize-environment-within-a-tabular-environment/154577
% 添加各类消息框
\newenvironment{rmdblock}[1]
  {
  \begin{table}[!htpb]
  \begin{tabular}{|cc|}
  \hline \quad \quad \quad &
  \begin{minipage}{.88\textwidth}
  \begin{itemize}
  \renewcommand{\labelitemi}{
    \raisebox{-.7\height}[0pt][0pt]{
      {\setkeys{Gin}{width=3em,keepaspectratio}\includegraphics{figure/bookdown/#1}}
    }
  }
  \setlength{\fboxsep}{1em}
  \begin{kframe}
  \item
  }
  {
  \end{kframe}
  \end{itemize}
  \end{minipage}
  \\
  \hline
  \end{tabular}
  \end{table}
  }
\newenvironment{rmdnote}
  {\begin{rmdblock}{note}}
  {\end{rmdblock}}
\newenvironment{rmdcaution}
  {\begin{rmdblock}{caution}}
  {\end{rmdblock}}
\newenvironment{rmdimportant}
  {\begin{rmdblock}{important}}
  {\end{rmdblock}}
\newenvironment{rmdtip}
  {\begin{rmdblock}{tip}}
  {\end{rmdblock}}
\newenvironment{rmdwarning}
  {\begin{rmdblock}{warning}}
  {\end{rmdblock}}

\begin{document}

$if(title)$
\title{$title$}
$endif$
$if(author)$
\author{$author$}
$endif$
$if(advisor)$
\advisor{$advisor$}
$endif$
$if(coadvisor)$
\coadvisor{$coadvisor$}
$endif$
$if(defenddate)$
\defenddate{$defenddate$}
$endif$
$if(school)$
\school{$school$}
$endif$
$if(institute)$
\institute{$institute$}
$endif$
$if(studentnumber)$
\studentnumber{$studentnumber$}
$endif$
$if(major)$
\major{$major$}
$endif$

$if(englishtitle)$
\englishtitle{$englishtitle$}
$endif$
$if(englishauthor)$
\englishauthor{\textsc{$englishauthor$}}
$endif$
$if(englishadvisor)$
\englishadvisor{Prof. \textsc{$englishadvisor$}}
$endif$
$if(englishcoadvisor)$
\englishcoadvisor{Prof. \textsc{$englishcoadvisor$}}
$endif$
$if(englishschool)$
\englishschool{$englishschool$}
$endif$
$if(englishinstitute)$
\englishinstitute{\textsc{$englishinstitute$} \\
  \textsc{Shanghai Jiao Tong University} \\
  \textsc{Shanghai, P.R.China}}
$endif$
$if(englishmajor)$
\englishmajor{$englishmajor$}
$endif$
$if(englishdate)$
\englishdate{$englishdate$}
$endif$
\maketitle

\makeatletter
\ifsjtu@submit\relax
	\includepdf{handed_pdf/original.pdf}
	\cleardoublepage
	\includepdf{handed_pdf/authorization.pdf}
	\cleardoublepage
\else
\ifsjtu@review\relax
% exclude the original claim and authorization
\else
	\makeDeclareOriginal
	\makeDeclareAuthorization
\fi
\fi
\makeatother

$body$

\backmatter	% 文后无编号部分 

%% 参考资料
\printbibliography[heading=bibintoc]

%% 致谢、发表论文、申请专利、参与项目、简历
%% 用于盲审的论文需隐去致谢、发表论文、申请专利、参与的项目
\makeatletter

%%
% "研究生学位论文送盲审印刷格式的统一要求"
% http://www.gs.sjtu.edu.cn/inform/3/2015/20151120_123928_738.htm

% 盲审删去删去致谢页
\ifsjtu@review\relax\else
  \include{tex/ack} 	  %% 致谢
\fi

\ifsjtu@bachelor
  % 学士学位论文要求在最后有一个英文大摘要,单独编页码
  \pagestyle{biglast}
  \include{tex/end_english_abstract}
\else
  % 盲审论文中,发表学术论文及参与科研情况等仅以第几作者注明即可,不要出现作者或他人姓名
  \ifsjtu@review\relax
    \include{tex/pubreview}
    \include{tex/projectsreview}  
  \else
    \include{tex/pub}	      %% 发表论文
    \include{tex/projects}  %% 参与的项目
  \fi
\fi

% \include{tex/patents}	  %% 申请专利
% \include{tex/resume}	  %% 个人简历

\makeatother

\end{document}
