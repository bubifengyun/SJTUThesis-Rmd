\documentclass[$if(fontsize)$$fontsize$,$endif$$if(lang)$$babel-lang$,$endif$$if(papersize)$$papersize$paper,$endif$$for(classoption)$$classoption$$sep$,$endfor$]{$documentclass$}

% 所有其它可能用到的包都统一放到这里了,可以根据自己的实际添加或者删除。
\usepackage{sjtuthesis}

% 定义图片文件目录与扩展名
\graphicspath{{figure/}}
\DeclareGraphicsExtensions{.pdf,.eps,.png,.jpg,.jpeg}

$for(bibliography)$
\addbibresource{$bibliography$}
$endfor$

\providecommand{\tightlist}{%
    \setlength{\itemsep}{0pt}\setlength{\parskip}{0pt}}

$if(href2footnote)$% 是否把链接改为脚注
\let\oldhref\href
\renewcommand{\href}[2]{#2\footnote{\url{#1}}}
$endif$

\makeatletter
% Setting Package listings/lstlistings
\lstset{tabsize=4, %
    aboveskip=\medskipamount,
    belowskip=\medskipamount,
    basicstyle=\footnotesize\ttfamily,
    commentstyle=\slshape\color{black!60},
    stringstyle=\color{green!40!black!100},
    keywordstyle=\bfseries\color{blue!50!black},
    extendedchars=false,
    upquote=true,
    tabsize=2,
    showstringspaces=false,
    xleftmargin=1em,
    xrightmargin=1em,
    breaklines=true,
    breakindent=2em,
    framexleftmargin=1em,
    framexrightmargin=1em,
    backgroundcolor=\color{gray!10},
    columns=flexible,
    keepspaces=true,
    texcl=true,
    mathescape=true
}
\makeatother

% 自定义命令



% https://github.com/jgm/pandoc/issues/4716
% lstinline 对数学模式存在 bug,等修复后可以改为如下方式
% \newcommand{\passthrough}[1]{#1}
\newcommand{\passthrough}[1]{\lstset{mathescape=false}#1\lstset{mathescape=true}} 

\lstset{mathescape=false}

% 解决双引号不一致的问题。在写完论文后,更新所有的双引号。单引号暂时没有发现问题。
\newcommand\cqh{“}
\newcommand\cqt{”}

% 添加引言块
\def\VA#1#2{\addvspace{12pt}\raggedleft #1\rightskip3em\par #2\rightskip3em}
\renewenvironment{quote}
  {\list{}{\rightmargin\leftmargin}%
    \item\relax}
  {\endlist}

$if(title)$
\title{$title$}
$endif$
$if(author)$
\author{$author$}
$endif$
$if(supervisor)$
\supervisor{$supervisor$}
$endif$
$if(cosupervisor)$
\cosupervisor{$cosupervisor$}
$endif$
$if(date)$
\date{$date$}
$endif$
$if(department)$
\department{$department$}
$endif$
$if(studentid)$
\studentid{$studentid$}
$endif$
$if(major)$
\major{$major$}
$endif$
$if(keywords)$
\keywords{$keywords$}
$endif$

$if(entitle)$
\entitle{$entitle$}
$endif$
$if(enauthor)$
\enauthor{\textsc{$enauthor$}}
$endif$
$if(ensupervisor)$
\ensupervisor{Prof. \textsc{$ensupervisor$}}
$endif$
$if(encosupervisor)$
\encosupervisor{Prof. \textsc{$encosupervisor$}}
$endif$
$if(endepartment)$
\endepartment{\textsc{$endepartment$}
$endif$
$if(enmajor)$
\enmajor{$enmajor$}
$endif$
$if(endate)$
\endate{$endate$}
$endif$
$if(enkeywords)$
\enkeywords{$enkeywords$}
$endif$

\begin{document}

\maketitle

\makeDeclareOriginality[handed_pdf/original.pdf]
\makeDeclareAuthorization

$body$

% 文后无编号部分
\backmatter

% 参考资料
\printbibliography[heading=bibintoc]

% 用于盲审的论文需隐去致谢、发表论文、参与项目、申请专利、简历

% 致谢
% !TEX root = ../thesis.tex

%TC:ignore

\begin{acknowledgements}
  感谢那位最先制作出博士学位论文 \LaTeX 模板的交大物理系同学!

  感谢 William Wang 同学对模板移植做出的巨大贡献!

  感谢 \href{https://github.com/weijianwen}{@weijianwen} 学长一直以来的开发和维
  护工作!

  感谢 \href{https://github.com/sjtug}{@sjtug} 以及
   \href{https://github.com/dyweb}{@dyweb} 对 0.9.5 之后版本的开发和维护工作!

  感谢所有为模板贡献过代码的同学们, 以及所有测试和使用模板的各位同学!

  感谢 \LaTeX 和 \href{https://github.com/sjtug/SJTUThesis}{\sjtuthesis},帮我节
  省了不少时间。
\end{acknowledgements}

%TC:endignore


% 发表论文、参与项目、申请专利、简历
% 盲审论文中,发表学术论文及参与科研情况等仅以第几作者注明即可,不要出现作者或他人姓名
% !TEX root = ../thesis.tex

%TC:ignore

\begin{publications}
  \item Chen H, Chan C~T. Acoustic cloaking in three dimensions using acoustic metamaterials[J]. Applied Physics Letters, 2007, 91:183518.
  \item Chen H, Wu B~I, Zhang B, et al. Electromagnetic Wave Interactions with a Metamaterial Cloak[J]. Physical Review Letters, 2007, 99(6):63903.
\end{publications}

\begin{publications*}
  \item 第一作者. 中文核心期刊论文, 2007.
  \item 第一作者. EI 国际会议论文, 2006.
\end{publications*}

%TC:endignore

% !TEX root = ../thesis.tex

%TC:ignore

\begin{projects}
  \item 参与973项目子课题(2007年6月--2008年5月)
  \item 参与自然基金项目(2005年5月--2005年8月)
  \item 参与国防项目(2005年8月--2005年10月)
\end{projects}

\begin{projects*}
  \item 973项目“XXX”
  \item 自然基金项目“XXX”
  \item 国防项目“XXX”
\end{projects*}

%TC:endignore

% !TEX root = ../thesis.tex

%TC:ignore

\begin{patents}
  \item 第一发明人,“永动机”,专利申请号202510149890.0
\end{patents}

\begin{patents*}
  \item 第一发明人,“永动机”,专利申请号XXXXXXXXXXXX.X
\end{patents*}

%TC:endignore

% !TEX root = ../thesis.tex

%TC:ignore

\begin{resume}
  \subsection*{基本情况}
    某某,yyyy 年 mm 月生于 xxxx。

  \subsection*{教育背景}
  \begin{itemize}
    \item yyyy 年 mm 月至今,上海交通大学,博士研究生,xx 专业
    \item yyyy 年 mm 月至 yyyy 年 mm 月,上海交通大学,硕士研究生,xx 专业
    \item yyyy 年 mm 月至 yyyy 年 mm 月,上海交通大学,本科,xx 专业
  \end{itemize}

  \subsection*{研究兴趣}
    \LaTeX{} 排版

  \subsection*{联系方式}
  \begin{itemize}
    \item 地址: 上海市闵行区东川路 800 号,200240
    \item E-mail: \email{xxx@sjtu.edu.cn}
  \end{itemize}
\end{resume}

%TC:endignore


% 中文学士学位论文要求在最后有一个英文大摘要,单独编页码,英文学士学位论文不需要
% !TEX root = ../thesis.tex

\begin{bigabstract}
  An imperial edict issued in 1896 by Emperor Guangxu, established Nanyang
  Public School in Shanghai. The normal school, school of foreign studies,
  middle school and a high school were established. Sheng Xuanhuai, the person
  responsible for proposing the idea to the emperor, became the first president
  and is regarded as the founder of the university.

  During the 1930s, the university gained a reputation of nurturing top
  engineers. After the foundation of People's Republic, some faculties were
  transferred to other universities. A significant amount of its faculty were
  sent in 1956, by the national government, to Xi'an to help build up Xi'an Jiao
  Tong University in western China. Afterwards, the school was officially
  renamed Shanghai Jiao Tong University.

  Since the reform and opening up policy in China, SJTU has taken the lead in
  management reform of institutions for higher education, regaining its vigor
  and vitality with an unprecedented momentum of growth. SJTU includes five
  beautiful campuses, Xuhui, Minhang, Luwan Qibao, and Fahua, taking up an area
  of about 3,225,833 m2. A number of disciplines have been advancing towards the
  top echelon internationally, and a batch of burgeoning branches of learning
  have taken an important position domestically.

  Today SJTU has 31 schools (departments), 63 undergraduate programs, 250
  masters-degree programs, 203 Ph.D. programs, 28 post-doctorate programs, and
  11 state key laboratories and national engineering research centers.

  SJTU boasts a large number of famous scientists and professors, including 35
  academics of the Academy of Sciences and Academy of Engineering, 95 accredited
  professors and chair professors of the "Cheung Kong Scholars Program" and more
  than 2,000 professors and associate professors.

  Its total enrollment of students amounts to 35,929, of which 1,564 are
  international students. There are 16,802 undergraduates, and 17,563 masters
  and Ph.D. candidates. After more than a century of operation, Jiao Tong
  University has inherited the old tradition of "high starting points, solid
  foundation, strict requirements and extensive practice." Students from SJTU
  have won top prizes in various competitions, including ACM International
  Collegiate Programming Contest, International Mathematical Contest in Modeling
  and Electronics Design Contests. Famous alumni include Jiang Zemin, Lu Dingyi,
  Ding Guangen, Wang Daohan, Qian Xuesen, Wu Wenjun, Zou Taofen, Mao Yisheng,
  Cai Er, Huang Yanpei, Shao Lizi, Wang An and many more. More than 200 of the
  academics of the Chinese Academy of Sciences and Chinese Academy of
  Engineering are alumni of Jiao Tong University.
\end{bigabstract}


\end{document}
